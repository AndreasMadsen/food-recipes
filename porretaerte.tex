\documentclass[a4paper]{article}

\usepackage[utf8]{inputenc}	% Flere sprog tegnsæt (fx æøå)
\usepackage[danish]{babel}	% Dansk orddeling (kan ændres til english)
\usepackage[T1]{fontenc}		% Brug 8-bit front
\usepackage{lmodern}		% Vektor front

\usepackage[svgnames]{xcolor} % Udvider \color med "SVG color names"
\usepackage{graphicx}	% Kompatibilitet til visning af pixel billeder (.png, .jpg, .gif)
\usepackage{epstopdf}	% Kompatibilitet til visning af vector billeder (.eps)
\usepackage{parskip}	% Tilføjer vertikal margin til hver paragraph
\usepackage{float}		% TIllader H som positions parameter
\usepackage{subcaption}	% Tillader subfigure, subtable samt \captions
\usepackage{amssymb}	% Flere matematiske symboler
\usepackage{amsthm}      % Endnu flere matematiske symboler
\usepackage{mathtools}	% Det meste matematik (indeholder ams­math og rettelser)
\usepackage{xfrac}		% Flere fracs (\sfrac{}{})
\usepackage{listings}	% Indsæt code
\usepackage{todonotes}	% Cool todo notes, [disable] skjuler todos
\usepackage[bookmarks,bookmarksnumbered,hidelinks]{hyperref} % clickable pdf (til sidst)

%listing settings, æøå support, font config, line number, left lines
\lstset{
    breakatwhitespace=false, breaklines=true,
    inputencoding=utf8, extendedchars=true,
    literate={å}{{\aa}}1 {æ}{{\ae}}1 {ø}{{\o}}1 {Å}{{\AA}}1 {Æ}{{\AE}}1 {Ø}{{\O}}1,
    keepspaces=true, showstringspaces=false, basicstyle=\small\ttfamily,
    frame=L, numbers=left, numberstyle=\scriptsize\color{gray},
    keywordstyle=\color{SteelBlue}\ttfamily,
    stringstyle=\color{IndianRed}\ttfamily,
    commentstyle=\color{Teal}\ttfamily,
} 

\setlength{\marginparwidth}{80pt} 				% Mere brede på margin notes og todos
\setlength{\parindent}{0cm}   					% Deaktiver afsnit indrykning
\DeclareGraphicsExtensions{.pdf,.eps,.png,.jpg,.gif}	% ændre til .png, .jpg for hurtig visning

\begin{document}

\subsection*{Bund}

\begin{itemize}
\item 250 g mel (graharmsmel eller fint speltmel)
\item 100 g smør
\item ½ tsk salt
\item 0,5 – 1 dl vand eller fløde
\item Evt et lille æg (men mindre vand)
\end{itemize}

Smørret smuldres ind i melet til det ligner revet ost.

Salt tilsættes.

Dejen samles hurtigt med vand/fløde og/eller æg. \textit{start med 0,5 dl væske}. Ælte ikke for meget, da det resultere i en sej dej.

Dejen trykkes direkte ud i formen. \textit{Den kan sættes på køl i ½ time og derefter rulle den ud, men er lettest at trykke ud før.} Prik den lidt med en gaffel eller kliv.

Tærtebunden forbages ved 225 grader i 10 min uden fyld.

\textit{Note: Tærtebunden kan godt bages nogle timer før anvendelse.}

\subsection*{Fyld}

\begin{itemize}
\item 4 Porrer (500 g)
\item 2 dl A-38
\item 3 æg
\item 1 pakke Kalkun bacon
\item Salt, peber
\item 50 g revet ost
\item Evt timian
\end{itemize}

Skær porrerne i tynde ringe og damp dem i lidt olie/smør til de er bløde.

Skær bacon i tern og svits dem på en pande.

Pisk A-38, æg, salt, peber, ost og evt. lidt timian sammen.

Rør bacon i æggeblandingen.

Fordel porrerne på den forbagte bund og hæld æggeblandingen over.

Tærten bages færdig ved 200 grader i ca. 30-35 minutter.

\end{document}